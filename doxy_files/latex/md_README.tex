\#\+Projeto Home\+Stark \+:coffee\+: 

 $\vert$ $\vert$ $\vert$ $\vert$ \+\_\+\+\_\+\+\_\+ \+\_\+ \+\_\+\+\_\+ \+\_\+\+\_\+\+\_\+ \+\_\+\+\_\+\+\_\+/ \+\_\+\+\_\+\+\_\+$\vert$$\vert$ $\vert$\+\_\+ \+\_\+\+\_\+ \+\_\+ \+\_\+ \+\_\+\+\_\+$\vert$ $\vert$ \+\_\+\+\_\+ $\vert$ $\vert$\+\_\+$\vert$ $\vert$/ \+\_\+ $|$ '\+\_\+ {\ttfamily \+\_\+ \textbackslash{} / \+\_\+ \textbackslash{}\+\_\+\+\_\+\+\_\+ \textbackslash{}$\vert$ \+\_\+\+\_\+/ \+\_\+} $\vert$ '\+\_\+\+\_\+$\vert$ $\vert$/ / $\vert$ \+\_\+ $\vert$ (\+\_\+) $\vert$ $\vert$ $\vert$ $\vert$ $\vert$ $\vert$ \+\_\+\+\_\+/\+\_\+\+\_\+\+\_\+) $\vert$ $\vert$$\vert$ (\+\_\+$\vert$ $\vert$ $\vert$ $\vert$ $<$ $\vert$\+\_\+$\vert$ $\vert$\+\_\+$\vert$\+\_\+\+\_\+\+\_\+/$\vert$\+\_\+$\vert$ $\vert$\+\_\+$\vert$ $\vert$\+\_\+$\vert$\+\_\+\+\_\+\+\_\+$\vert$\+\_\+\+\_\+\+\_\+\+\_\+/ \+\_\+\+\_\+\+\_\+\+\_\+,\+\_\+$\vert$\+\_\+$\vert$ $\vert$\+\_\+$\vert$\+\_\+\textbackslash{}

$>$make T\+A\+R\+G\+E\+T=srf06-\/cc26xx

Desenvolvedor\+: Ânderson Ignácio da Silva

Projeto de T\+C\+C para criação de uma rede mesh 6\+Lo\+W\+P\+A\+N utilizando o target cc2650 da Texas Instruments. O dispositivo será capaz de se conectar a uma rede mesh 6\+Lo\+W\+P\+A\+N, comunicando via M\+Q\+T\+T-\/\+S\+N com broker remoto e local através de um interface de gestão/configuração.

Características\+:
\begin{DoxyItemize}
\item \mbox{[} \mbox{]} Suporte completo a M\+Q\+T\+T-\/\+S\+N
\item \mbox{[} \mbox{]} D\+T\+L\+S sobre U\+D\+P
\item \mbox{[} \mbox{]} Informação ao border router sobre o nó
\item \mbox{[} \mbox{]} Comunicação com periféricos e afins
\item \mbox{[} \mbox{]} Documentação em Doxygen
\end{DoxyItemize}

Nomenclaturas\+:

E\+T\+X (expected transmission count) = Medidor de qualidade de caminho entre dois nós em um pacote wireless de rede. Basicamente esse núme ro indica o número esperado de transmissões de um pacote necessária s para que não haja erro na recepção no destino. 