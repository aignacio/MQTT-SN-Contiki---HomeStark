\#\+Porte do M\+Q\+T\+T-\/\+S\+N para o Contiki-\/\+O\+S  {\bfseries Desenvolvedor\+:} Ânderson Ignácio da Silva  Portado parcialmente as características oficiais do protocolo, alguns limitantes ainda são impostos pelo broker (mosquitto.\+rsmb).  {\bfseries Características\+:}
\begin{DoxyItemize}
\item \mbox{[}x\mbox{]} publicação e inscrição com Qo\+S 0
\item \mbox{[}x\mbox{]} wildcard (ver observações abaixo)
\item \mbox{[}x\mbox{]} função de callback para recebimento de mensagens
\item \mbox{[}x\mbox{]} auto-\/reconexão
\item \mbox{[}x\mbox{]} documentação em doxygen
\item \mbox{[}x\mbox{]} teste em simulação e cross-\/compile
\item \mbox{[} \mbox{]} suporte a Qo\+S -\/1, 1 e 2 
\end{DoxyItemize}

{\bfseries Iniciando\+:}  Inicialmente recomenda-\/se testar com o broker em anexo (tools/mosquitto.\+rsmb) para o iniciar o broker M\+Q\+T\+T-\/\+S\+N com {\bfseries I\+Pv6} habilitado\+: ```make ./broker\+\_\+mqtts config.\+mqtt ``` Uma vez iniciado pode-\/se conectar tanto via {\bfseries T\+C\+P (porta 1883)} ou {\bfseries U\+D\+P (porta 1884)}, recomenda-\/se manter uma janela de terminal aberta com inscrição no tópico \# utilizando recursos do mosquitto para visualizar quaisquer pacotes trocados no broker\+: ```make mosquitto\+\_\+sub -\/t \char`\"{}\#\char`\"{} -\/v ``` {\bfseries Compilando\+:}  Para testar o porte foram realizados testes na ferramenta {\bfseries cooja} com o mote Z1, e com um teste real com o C\+C2650. Para utilizar a simulação, abra o cooja dentro de contiki/tools/cooja\+: ```make ant run ``` E abra o a simulação dentro de tools/simulacoes (mqtt\+\_\+sn\+\_\+exemplo) para testar compile para o {\bfseries T\+A\+R\+G\+E\+T=z1}. É necessário que este repositório esteja dentro da pasta do contiki para encontrar os diretórios no makefile. ```make make T\+A\+R\+G\+E\+T=z1 ``` Algumas flags foram adicionadas no {\bfseries Makefile} para reduzir o tamanho do firmware gerado. ```make C\+F\+L\+A\+G\+S += -\/ffunction-\/sections L\+D\+F\+L\+A\+G\+S += -\/\+Wl,--gc-\/sections,--undefined=\+\_\+reset\+\_\+vector\+\_\+\+\_\+,--undefined=Interrupt\+Vectors,--undefined=\+\_\+copy\+\_\+data\+\_\+init\+\_\+\+\_\+,--undefined=\+\_\+clear\+\_\+bss\+\_\+init\+\_\+\+\_\+,--undefined=\+\_\+end\+\_\+of\+\_\+init\+\_\+\+\_\+ ``` Para criar o tunelamento dos pacotes de simulação deve-\/se utilizar o script {\bfseries webserver\+\_\+slip.\+sh} (scripts\+\_\+aux) assim que a simulação estiver aberta, logo ao criar o adaptador de rede tunelado os pacotes serão redirecionados para o broker local do mosquitto.\+rsmb.  

{\bfseries Teste com dispositivo real(cc2650)\+:}

Para teste com o cc2650 a biblioteca irá adicionar o arquivo {\bfseries lib/newlib/syscalls.\+c} o qual é utilizado para alocação dinâmica de memória no heap do micro processador (malloc) através da chamada sbrk do sistema. Um dos detalhes percebidos é de que até o momento deste código o linker script do C\+C2650 no Contiki não possui definição de zona de heap, entretanto para utilizar de funções que utilizem desta zona de memória, insira no linker script do Contiki (contiki/cpu/cc26xx-\/cc13xx/cc26xx.\+ld) as linhas abaixo, no final da chave S\+E\+C\+T\+I\+O\+N\+S\+: ``` ... \+\_\+heap = .; \+\_\+eheap = O\+R\+I\+G\+I\+N(\+S\+R\+A\+M) + L\+E\+N\+G\+T\+H(\+S\+R\+A\+M); \} ``` A compilação para esse plataforma se dá através do {\bfseries T\+A\+R\+G\+E\+T=srf06-\/cc26xx} para programação também se utiliza da ferramenta uniflash, a qual deve ser previamente instalada caso haja interesse em programar dispositivos Texas. Como auxílio para a programação e teste há dois shell scripts (program\+C\+C2650.\+sh e script\+\_\+init\+\_\+slip.\+sh) onde o primeiro é utilizado para programação do dispositivo através do uniflash tool e o segundo é utilizado para iniciar o tunelamento de porta, o qual cria o dispositivo tun0 no linux, utilizado pelo border-\/router para redirecionar os pacotes enviados e recebidos da rede mesh. Também há um terceiro script (reset\+\_\+acm.\+sh) o qual resseta o módulo do linux utilizado pela conexão dos dispositivos do cc caso a enumeração ultrapasse valores máximo de alocação do sistema. Todos os scripts estão contidos dentro de {\bfseries scripts\+\_\+aux}. O target default do makefile do projeto é o cc2650 e para listar demais targets deve-\/se utilizar o comando \char`\"{}make targets\char`\"{}. Abaixo segue a sequência de comandos para compilar, programar e testar com o C\+C2650. Inicia-\/se pela compilação do rpl-\/border-\/router\+: ```make cd contiki/examples/ipv6/rpl-\/border-\/router make T\+A\+R\+G\+E\+T=srf06-\/cc26xx cd contiki contiki/\+M\+Q\+T\+T-\/\+S\+N-\/\+Contiki---Home\+Stark/scripts\+\_\+aux/program\+C\+C2650.\+sh border-\/router.\+srf06-\/cc26xx \#\+Aguardar a programação \#\+Abrir um novo terminal com broker já aberto(./broker\+\_\+mqtts config.\+mqtt) \#\+Iniciar o tunelamento (criação do tun0) contiki/\+M\+Q\+T\+T-\/\+S\+N-\/\+Contiki---Home\+Stark/scripts\+\_\+aux/script\+\_\+init\+\_\+slip.\+sh /dev/tty\+A\+C\+M\+X \#descobrir qual dispositivo está listado ls /dev/tty$\ast$, pegar o de menor valor \#Às vezes recomenda-\/se ressetar a placa do slip caso não aparece o ip atribuído ``` Em um novo terminal, programe o nó com o exemplo padrão\+: ```make cd contiki/\+M\+Q\+T\+T-\/\+S\+N-\/\+Contiki---Home\+Stark make T\+A\+R\+G\+E\+T=srf06-\/cc26xx all scripts\+\_\+aux/program\+C\+C2650.\+sh main\+\_\+core.\+elf \#\+Aguardar a programação ``` Assim que o dispositivo for programado ele tentará conectar ao broker e em sequência irá enviar as mensagens direcionadas conforme o exemplo, o tempo de conexão varia, algo em torno de 10 segundos aproximadamente.

\subsection*{Observações}


\begin{DoxyEnumerate}
\item Se o C\+C2650 utilizado for o launchpad, ele pode apresentar erro na programação com o Uniflash Tool, tal problema relacionado com o programador J\+T\+A\+G anexado a placa, para resolver isso, conecte o launchpad a um Windows com o software \href{http://software-dl.ti.com/dsps/forms/self_cert_export.html?prod_no=flash-programmer-2-1.7.4.zip&ref_url=http://software-dl.ti.com/lprf/flash_programmer_2}{\tt Smart\+R\+F Flash Programmer v2} e clique em \char`\"{}update\char`\"{} que após o update da ferramenta, ele irá ser programado no Linux.
\item Toda documentação das funções está em doxygen no caminho M\+Q\+T\+T-\/\+S\+N-\/\+Contiki---Home\+Stark/doxy\+\_\+files/html/index.\+html.
\end{DoxyEnumerate}

\subsection*{Contribuições e licença\+:}

Este software está sendo liberado sobre a licença Apache 2.\+0, qualquer contribuição deve ser informada ao autor, criando um branch novo para o feature implementado. 